\chapter*{Introduction}
\addcontentsline{toc}{chapter}{\textbf{Introduction}}
\markright{INTRODUCTION}

\qquad The evolution of modern technologies has lead to an increased number of Do-It-Yourself $(DIY)$ possibilities in
the field of electronics and microchips; thus it has become easy and affordable to incorporate such
technologies in mundane, house-hold objects that can present themselves with some data that may be relevant.
This new concept turned into a movement now called the Internet of Things $(IoT)$.

We use the term “Internet of Things” to refer to the general idea of things, especially everyday objects, that
are readable, recognizable, locatable, addressable, and/or controllable via the Internet,
wireless LAN, wide-area network, or other means. Everyday objects includes not only the electronic devices we
encounter everyday, and not only the products of higher technological development such as vehicles and
equipment, but things that we do not ordinarily think of as electronic at all, such as food, clothing, and
shelter; materials, arts, and sub-assemblies; commodities and luxury items; landmarks, boundaries,
and monuments; and all the miscellany of commerce and culture. \cite{DisruptiveCivilTechnologies}

In order for this concept to grow, technologies had to be made available to the general public, meaning that
the cost at which they came had to be affordable. Fortunately, the current technological context lead to
the development of cheap and powerful components which can now be used to build the infrastructure of the IoT.

With the now given possibility to acquire powerful, yet cheap, means by which one can control and gather data
from different contexts, various applications can be developed, like the one described by this paper.

The field of Internet of Things has known a major boom in the recent years, adding to its infrastructure
all kinds of objects, appliances etc, giving the possibility to monitor and analyze data from just about
anything. The broader affect of this, is the fact that data can be stored, analyzed and made sense of it
in order to increase the comfort level and to grasp a better understanding of the environment we are living in.

Starting from this idea, this paper describes a new application that makes uses of the cheap technology we
have at hand and the easy access to a world wide data transport infrastructure (the Internet). The result is
a device that increases home comfort levels, gathers data that is analyzed and makes of it information that
is digestible by a human being, enabling one to better understand and control the context one is living in.

The chosen hardware for the proposed thermostat consist of 3 micro-controllers, temperature and humidity sensors
and radio transceivers, all coming to a price comparable to a normal thermostat. The main processing unit
is an ESP8266 wireless module. An in depth description of the architecture will be presented in
section.
