\chapter*{Abstract}
\thispagestyle{empty}
\addtocounter{page}{-1}

\scriptsize

\qquad This paper presents an accessible approach to an efficient, intelligent, Internet of Things $(IoT)$ enabled
thermostat to be used, instead of the usual, naive thermostats that are currently in a reasonable price range
on the market. The current means by which most of the population control their house temperature is to use a
simple, straight-forward thermostat that is simply reactive to only the temperature it records and the
temperature set point of the user. The time we live in allows us to consider new approaches to temperature
control that greatly reduce the production costs and enables greater control over how we choose to spend energy
in this direction.

\qquad With the help of fuzzy logic and the microcontrollers we now have at hand, it is possible
to create a thermostat that is first of all affordable, accessible through the Internet and most of all
intelligent, adapting to its environment based on o larger number of factors such as humidity and the rate at
which the temperature rises or decreases. The presented application uses two types of microcontrollers, the ESP8266
that has 802.11 b/g/n wireless capabilities and a full TCP and UDP stack built on it and the Attiny85
developed by Atmel. The ESP8266 provides the memory and computational power to drive the fuzzy logic engine
while the Attiny85 is used to perform tasks like actuating a relay, act as an interface between the ESP8266
and a radio receiver, and collect data from sensors and send it through radio signals.
The general architecture follows an agent-like pattern, having a central unit (ESP8266), where the fuzzy engine and user
interaction software (low level webserver) is running and one or more \say{reporters} which are placed around the environment.
The \say{reporters} have the task to read temperature and humidity data and sent it to the central unit for
processing. The thermostat has the capability to publish its data on a remote server, where more complex
processing may take place. It can also receive commands over the Internet, enabling one to control it in a
remote fashion.


\qquad This paper aims to achieve three main goals. First of all, to demonstrate that with the currently
available hardware, an Internet of Things enabled thermostat can be build that would go into a price range
comparable to normal thermostats. By normal, we mean thermostats that cannot be accessed through the Internet
and do not have any intelligent decision making capabilities. Secondly, we aim to show that nowadays we have
sufficiently powerful microcontrollers that can perform advanced enough tasks to make them suitable
for an intelligent, fuzzy logic-based thermostat. Thirdly, this paper aims to show that having a thermostat
that is integrated into the Internet of Things does not have to be an expensive choice and that is should be
a rather ordinary feature.

\qquad Results have shown that taking this approach at temperature control makes for a thermostat that is not
only cheaper, but also smarter, being able to make decisions that are based on broader range of factors which
lead to a more efficient energy consumption.

\qquad The anticipated outcome of this approach is to create a new kind of
thermostat that is first of all affordable, making the most of the new technologies we have, accessible through
the infrastructure which the Internet provides and intelligent, giving the user an improved experience and
greater energy management control.

\qquad This work is the result of my own activity. I have neither given nor received unauthorized
assistance on this work.

\begin{flushright}
Candale Andrei
\end{flushright}


\normalsize
