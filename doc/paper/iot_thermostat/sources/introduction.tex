\chapter*{Introduction}
\addcontentsline{toc}{chapter}{\textbf{Introduction}}
\markright{INTRODUCTION}

\qquad The evolution of modern technologies has lead to an increased number of Do-It-Yourself $(DIY)$ possibilities in
the field of electronics and microchips; thus it has become easy and affordable to incorporate such
technologies in mundane, house-hold objects that can present themselves with some data that may be relevant.
This new concept turned into a movement now called the Internet of Things $(IoT)$.

\qquad We use the term “Internet of Things” to refer to the general idea of things, especially everyday objects, that
are readable, recognizable, locatable, addressable, and/or controllable via the Internet,
wireless LAN, wide-area network, or other means. Everyday objects includes not only the electronic devices we
encounter everyday, and not only the products of higher technological development such as vehicles and
equipment, but things that we do not ordinarily think of as electronic at all, such as food, clothing, and
shelter; materials, arts, and sub-assemblies; commodities and luxury items; landmarks, boundaries,
and monuments; and all the miscellany of commerce and culture. \cite{DisruptiveCivilTechnologies}

In order for this concept to grow, technologies had to be made available to the general public, meaning that
the cost at which they came had to be affordable. Fortunately, the current technological context lead to
the development of cheap and powerful components which can now be used to build the infrastructure of the IoT.

With the now given possibility to acquire powerful, yet cheap, means by which one can control and gather data
from different contexts, various applications can be developed, like the one described by this paper.

\qquad The field of Internet of Things has known a major boom in the recent years, adding to its infrastructure
all kinds of objects, appliances etc, giving the possibility to monitor and analyze data from just about
anything. The broader affect of this, is the fact that data can be stored, analyzed and made sense of it
in order to increase the comfort level and to grasp a better understanding of the environment we are living in.

Fuzzy logic was first proposed by Lotfi Zadef in his paper \say{Fuzzy Sets} \cite{FuzzySets}. At first, the
idea received a lot of criticism from the mathematical community and the world was skeptic in implementing it
on real devices, but Japan took the lead and the first real-life applications that used fuzzy logic appeared on
the market in the late 1980's \cite{zadeh1996fuzzy}. The most memorable such application was the Sendai subway
system designed by Hitachi where fuzzy logic was used to control the train's acceleration, deceleration and
breaking \cite{zadeh1996fuzzy}. Since its production it is said that it not only reduces the energy
consumption by 10\%but the passengers hardly notice when the train is changing its velocity
\cite{FuzzyLogicMassUse}. Since then, fuzzy logic has been used in devices ranging from shower heads
\cite{zadeh1996fuzzy} to gear boxes and car breaking systems \cite{FuzzyAutomotive}.

In the field of temperature control, fuzzy logic was applied with success in AC's \cite{FuzzyAppliences} where
an energy saving of at least 3.5\% was observed. It has been shown that by this approach at temperature
control, given a small number of rules and not needing advanced knowledge about the environment (e.g. a
mathematical model of the problem is not required) a fuzzy logic controller is able to achieve faster transient
response with less overshoot, more stable steady state response and less dependence of operating point
\cite{FuzzyTempControl}.

This paper presents an approach to a fuzzy logic thermostat that aims to increase the energy efficiency
of temperature control, where a standard, ON/OFF, central heating unit is used. An important aspect of this
proposal is the hardware implementation which is designed to reduce the cost of such an application and to
enable it to be accessed through the infrastructure of the Internet.

\qquad The thesis is structured in five chapters, as follows. \par
\qquad Chapter I, \textbf{Theoretical background} presents to the reader the concepts surrounding fuzzy logic
and the Internet of Things. It first addresses the building block of fuzzy logic, presenting it through
comparison with the classical logic and then presents a short introduction to the Internet of Things

\qquad Chapter II, \textbf{Our approach}, introduces the reader to the specifics of the fuzzy engine
that the proposed thermostat is using. More precisely, the motivation behind the thermostat is presented, the
variables that the fuzzy engine takes, the rules that make up its knowledge and how the engine arrives at a
conclusion.This chapter is an continuation of our previous article \cite{FuzzyLogicIOT}.

\qquad Chapter III, \textbf{Results and discussions}, presents the results that were obtained while testing
both the hardware and the software.

\qquad Chapter IV, \textbf{Application}, describes the hardware and software architecture, the electronics
and microcontrollers that are used, the API's of chips on which the software runs and how all of the
components come together to form the thermostat.

\quad Chapter V, \textbf{Conclusions}, presents to the reader the conclusions that were drawn from developing
the described application.

\section*{List of publications:}

\begin{itemize}
    \item \textbf{Candale Andrei}. Fuzzy logic-based IoT enabled thermostat. Sesiunea de Comunicari
    Stiintifice ale Studentilor, Cluj-Napoca (Romania), June 19, 2015. --- accepted
\end{itemize}
