\documentclass[conference]{IEEEtran}

\usepackage[section]{algorithm}
\usepackage{algorithmic}
\usepackage{enumerate}
\usepackage{mathtools}
\usepackage{tablefootnote}
\usepackage{color}

\begin{document}

\title{Fuzzy Logic, IoT enabled thermostat}

\author{\IEEEauthorblockN{Candale Andrei}
\IEEEauthorblockA{Department of Computer Science\\
Babe\c{s}-Bolyai University\\
1 M. Kogalniceanu Street, 400084, Cluj-Napoca, Romania\\
Email: caie1378@scs.ubbcluj.ro}}

\maketitle


\begin{abstract}

This paper presents an accessible approach to an efficient, intelligent, Internet of Things (IoT) enabled
thermostat to be used, instead of the usual, naive thermostats that are currently in a resonable price range
on the market. The currentmeans by which most of the population control their house temperature is to use a
simple, straight-forward thermostat that is simply reactive to only the temperature it records and the
temperature set point of the user. The goal is to show that with the current technologies we have at hand,
we can build a thermostat that is accessible through the infrastructure that the Internet provides and that
can adapt in an intelligent manner condering more than just the temperature error. With the advent of new,
afordable technologies, the possibily of builing such a device has rocketed and with the advent of the new
ESP8266 wireless module, making a IoT enabled device became suddently an appealing and obvious option. The
anticipated outcome of this approach is to create a new kind of thermostat that is first of all afordable,
making the most of the new technologies we have, and intelliget, giving the user an improved experience and
greater energy management control.

\end{abstract}


\section{Introduction}
\label{Introduction}



\section{Motivation}
\label{Motivation}


\section{Related work}
\label{Related work}


\section{Theoretical Background}
\label{Theoretical Background}


\section{Methodology}
\label{Methodology}


\section{Experiments}
\label{Experiments}



\section{Conclusion}
\label{Conclusion}


\bibliographystyle{IEEEtran}
\bibliography{IEEEabrv,bibliography}

\end{document}
